\documentclass[book, nodocumentinfo]{upmethodology-document}

%% The TeX code is entering with UTF8
%% character encoding (Linux and MacOS standards)
\usepackage[utf8]{inputenc}
%% For algorithms
\usepackage[chapter]{algorithm}
\usepackage{algpseudocode}
%% For bibtex
\usepackage{natbib}

\setfrontcover{classic}

\declaredocument{Report about Radix Sort LO27 Project}{Report about Radix Sort LO27 Project for the UTBM}{LO27-A2015}
\setpublisher{University of Technology of Belfort-Montbéliard}

\incversion{\makedate{7}{11}{2015}}{Initial version.}{\upmpublic}

\addauthorvalidator*[benoit.cortier@utbm.fr]{Benoît}{CORTIER}{Author}
\addauthorvalidator*[jerome.boulmier@utbm.fr]{Jérôme}{BOULMIER}{Author}

\setdockeywords{\LaTeX, Radix Sort, Sorting, Algorithm, LO27, Programming}

\setdocabstract{}

\makeatletter
\let\VERversion\upm@package@version@ver
\let\VERfmt\upm@package@fmt@ver
\let\VERdoc\upm@package@doc@ver
\let\VERfp\upm@package@fp@ver
\let\VERbp\upm@package@bp@ver
\let\VERext\upm@package@ext@ver
\let\VERtask\upm@package@task@ver
\let\VERdocclazz\upm@package@docclazz@ver
\let\VERcode\upm@package@code@ver
\makeatother

\setcopyrighter{Benoît CORTIER \& Jérôme BOULMIER}
\setprintingaddress{France}

% Espace insécable (le caractère ~ se traduit par un espace insécable).
\DeclareUnicodeCharacter{00A0}{~}
% Sur un clavier bépo, on a un accès direct au ×, ≥ et ≤
\DeclareUnicodeCharacter{00D7}{\times}
\DeclareUnicodeCharacter{2265}{\geq}
\DeclareUnicodeCharacter{2264}{\geq}

\newcommand{\separator}{\centerline{********************}}

\begin{document}

\chapter{Introduction}

The goal of this project is to provide a radix sort algorithm and its C implementation
to sort a list of N integers expressed in a base B between 2 and 16.

The radix sort is a method to sort a list of integers represented in any base.
This method is non-comparative: numbers are not compared to each other.
There are two classifications: \emph{least significant digit} (LSD) or leftmost radix sorts
and \emph{most significant digit} (MSB) or rightmost radix sorts.
LSD radix sorts process the integer representation from rightmost digit to leftmost digit
while MSD radix sorts work the other way around.
We will provide an algorithm for an LSD radix sort.
The principle is:
\begin{itemize}
    \item Numbers are distributed into B buckets according to the rightmost
        digit (the digit have B different possible values).
    \item Then buckets are connected in ascending order.
    \item The same operation is applied using the second rightmost digit and so on.
    \item Once the operation has been applied to the leftmost digit, the list is sorted.
\end{itemize}

This report is structured as follows.
Chapter \ref{chapter:objectives-problem} presents the project goals and problem statements.
Chapter \ref{chapter:data-structures} defines the different data structures that we used to design algorithms
and to make the C implemention.
Chapter \ref{chapter:algorithms} provides the various algorithms ready to be implemented in any programming language.

\tableofcontents

\listofalgorithms

\chapter{Objectives and problem statements} \label{chapter:objectives-problem}

…

\chapter{Data structures} \label{chapter:data-structures}

…

\section{BaseNIntegerList}

The type \emph{BaseNIntegerList} represents a list of base-N integers.
This list will be implemented using a doubly linked-list.

\begin{table}[h]
    \centering
    \label{tab:basenintegerlist-struct}

    \begin{tabular}{|l|c|}
        \hline
        BaseNIntegerElement* & head \\
        \hline
        BaseNIntegerElement* & tail \\
        \hline
        int & base \\
        \hline
        int & size \\
        \hline
    \end{tabular}

    \caption{BaseNIntegerList structure}
\end{table}

\begin{itemize}
    \item \emph{head}: pointer to the first element of the list.
    \item \emph{tail}: pointer to the last element of the list.
    \item \emph{base}: an integer between 2 and 16.
    \item \emph{size}: the current number of elements in the list.
\end{itemize}

\subsection{BigInteger}

\begin{table}[h]
    \centering
    \label{tab:basenintegerelement-big-integer}

    \begin{tabular}{|l|c|}
        \hline
        char* & value \\
        \hline
        int & size \\
        \hline
    \end{tabular}

    \caption{BigInteger structure}
\end{table}

\begin{itemize}
    \item \emph{value}: the integer defined by an array of characters.
    \item \emph{size}: the size of the integer (number of digits).
\end{itemize}

\subsection{BaseNIntegerElement}

\begin{table}[h]
    \centering
    \label{tab:basenintegerelement-struct}

    \begin{tabular}{|l|c|}
        \hline
        BaseNIntegerElement* & prev \\
        \hline
        BaseNIntegerElement* & next \\
        \hline
        BigInteger & value \\
        \hline
    \end{tabular}

    \caption{BaseNIntegerElement structure}
\end{table}

\begin{itemize}
    \item \emph{prev}: pointer to the predecessor.
    \item \emph{next}: pointer to the successor.
    \item \emph{value}: the integer represented in a base between 2 and 16.
\end{itemize}

\section{BaseNIntegerListOfList}

\begin{table}[h]
    \centering
    \label{tab:basenintegerlistoflist-struct}

    \begin{tabular}{|l|c|}
        \hline
        Integer & base \\
        \hline
        BaseNIntegerList* & array \\
        \hline
    \end{tabular}

    \caption{BaseNIntegerListOfList structure}
\end{table}

\chapter{Algorithms} \label{chapter:algorithms}

…

\section{BigInteger}

\emph{copyBigInteger}: \emph{BigInteger} \(\rightarrow \emptyset\),
    returns a copy of the given \emph{BigInteger}.
    \ref{algo:basenintegerlist-copy-big-integer} is the corresponding algorithm.

\begin{algorithm}[H]
    \caption{copyBigInteger algorithm}
    \label{algo:basenintegerlist-copy-big-integer}

    \begin{algorithmic}
        \Function{copyBigInteger}{intToCopy : \emph{BigInteger}} : \(\emptyset\)
        \State value \(\leftarrow createArray<Character>(size(intToCopy))\)
            \State memcpy(value, value(intToCopy), size(intToCopy) \(×\) sizeof(Character))
            \State copyBigInteger \(\leftarrow\) createBigInteger(value, size(intToCopy))
        \EndFunction
    \end{algorithmic}
\end{algorithm}

Assumption: \emph{sizeof} returns the size taken in memory by the given type.

\section{BaseNIntegerList}

\emph{createIntegerList}: \emph{Integer} \(\rightarrow\) \emph{BaseNIntegerList},
    creates a new empty \emph{BaseNIntegerList} for storing integers in the specified base.
    \ref{algo:basenintegerlist-create-integer-list} is the corresponding algorithm.

\begin{algorithm}[H]
    \caption{createIntegerList algorithm}
    \label{algo:basenintegerlist-create-integer-list}

    \begin{algorithmic}
        \Function{createIntegerList}{base : Integer} : BaseNIntegerList
            \State \(l \leftarrow\) create(BaseNIntegerList)
            \State head(\(l\)) \(\leftarrow\) undefined
            \State tail(\(l\)) \(\leftarrow\) undefined
            \State base(\(l\)) \(\leftarrow\) base
            \State size(\(l\)) \(\leftarrow 0\)
            \State createIntegerList \(\leftarrow l\)
        \EndFunction
    \end{algorithmic}
\end{algorithm}

\separator

\emph{isEmpty}: \emph{BaseNIntegerList} \(\rightarrow\) \emph{Boolean},
returns true if the specified list is empty, false otherwise.
\ref{algo:basenintegerlist-is-empty} is the corresponding algorithm.

\begin{algorithm}[H]
    \caption{isEmpty algorithm}
    \label{algo:basenintegerlist-is-empty}

    \begin{algorithmic}
        \Function{isEmpty}{\(l\) : BaseNIntegerList} : Boolean
            \If{head(\(l\)) = undefined}
                \State isEmpty \(\leftarrow\) TRUE
            \Else
                \State isEmpty \(\leftarrow\) FALSE
            \EndIf
        \EndFunction
    \end{algorithmic}
\end{algorithm}

\separator

\emph{insertHead}: \emph{BaseNIntegerList} \(×\) \emph{BigInteger} \(\rightarrow\) \emph{BaseNIntegerList},
adds the specified integer (\emph{BigInteger}, represented in the considered base) at the beginning
of the specified list.
\ref{algo:basenintegerlist-insert-head} is the corresponding algorithm.

\emph{Principle}
\begin{itemize}
    \item create a new element.
    \item link it and the rest of the list to each other if any.
    \item make it the head of the list.
    \item increase size of the list accordingly.
\end{itemize}

\begin{algorithm}[H]
    \caption{insertHead algorithm}
    \label{algo:basenintegerlist-insert-head}

    \begin{algorithmic}
        \Function{insertHead}{\(l\) : \emph{BaseNIntegerList}, n : \emph{BigInteger}} : \emph{BaseNIntegerList}
            \State new\_el \(\leftarrow\) create(BaseNIntegerElement)

            \State value(new\_el) \(\leftarrow\) n
            \State prev(new\_el) \(\leftarrow\) undefined

            \If{isEmpty(\(l\))}
                \State next(new\_el) \(\leftarrow\) undefined
                \State tail(\(l\)) \(\leftarrow\) new\_el
            \Else
                \State next(new\_el) \(\leftarrow\) head(\(l\))
                \State prev(head(\(l\))) \(\leftarrow\) new\_el
            \EndIf

            \State head(\(l\)) \(\leftarrow\) new\_el
            \State size(\(l\)) \(\leftarrow\) size(\(l\)) + 1

            \State insertHead \(\leftarrow l\)
        \EndFunction
    \end{algorithmic}
\end{algorithm}

\separator

\emph{insertTail}: \emph{BaseNIntegerList} \(×\) \emph{BigInteger} \(\rightarrow\) \emph{BaseNIntegerList},
adds the specified integer (\emph{BigInteger}, represented in the considered base) at the end
of the specified list.
\ref{algo:basenintegerlist-insert-tail} is the corresponding algorithm.

\emph{Principle}
\begin{itemize}
    \item create a new element.
    \item link it and the tail of the list to each other if any.
    \item make it the tail of the list.
    \item increase size of the list accordingly.
\end{itemize}

\begin{algorithm}[H]
    \caption{insertTail algorithm}
    \label{algo:basenintegerlist-insert-tail}

    \begin{algorithmic}
        \Function{insertTail}{\(l\) : BaseNIntegerList, n : \emph{BigInteger}} : BaseNIntegerList
            \State new\_el \(\leftarrow\) create(BaseNIntegerElement)

            \State value(new\_el) \(\leftarrow\) n
            \State next(new\_el) \(\leftarrow\) undefined

            \If{isEmpty(\(l\))}
                \State prev(new\_el) \(\leftarrow\) undefined
                \State head(\(l\)) \(\leftarrow\) new\_el
            \Else
                \State prev(new\_el) \(\leftarrow\) tail(\(l\))
                \State next(tail(\(l\))) \(\leftarrow\) new\_el
            \EndIf

            \State tail(\(l\)) \(\leftarrow\) new\_el
            \State size(\(l\)) \(\leftarrow\) size(\(l\)) + 1

            \State insertTail \(\leftarrow l\)
        \EndFunction
    \end{algorithmic}
\end{algorithm}

\separator

\emph{removeHead}: \emph{BaseNIntegerList} \(\rightarrow\) \emph{BaseNIntegerList},
removes the first element of the specified list.
\ref{algo:basenintegerlist-remove-head} is the corresponding algorithm.

\emph{Principle}
\begin{itemize}
    \item makes the second element the head of the list if any.
    \item free the first element from the memory.
    \item decrease size of the list accordingly.
\end{itemize}

\begin{algorithm}[H]
    \caption{removeHead algorithm}
    \label{algo:basenintegerlist-remove-head}

    \begin{algorithmic}
        \Function{removeHead}{\(l\) : BaseNIntegerList} : BaseNIntegerList
            \If{isEmpty(l)}
                \State removeHead \(\leftarrow l\)
            \EndIf

            \State BaseNIntegerElement old\_el \(\leftarrow\) head(l)
            \State head(l) \(\leftarrow\) next(head(l))
            \State deleteBaseNIntegerElement(old\_el)
            \State size(\(l\)) \(\leftarrow\) size(\(l\)) - 1

            \If{isEmpty(l)}
                \State tail(l) \(\leftarrow\) undefined
            \Else
                \State prev(head(l)) \(\leftarrow\) undefined
            \EndIf

            \State removeHead \(\leftarrow l\)
        \EndFunction
    \end{algorithmic}
\end{algorithm}

\separator

\emph{removeTail}: \emph{BaseNIntegerList} \(\rightarrow\) \emph{BaseNIntegerList},
removes the last element of the specified list.
\ref{algo:basenintegerlist-remove-tail} is the corresponding algorithm.

\emph{Principle}
\begin{itemize}
    \item makes element n-1 the tail of the list if n \(>\) 1.
    \item free element n from the memory.
    \item decrease size of the list accordingly.
\end{itemize}

\begin{algorithm}[H]
    \caption{removeTail algorithm}
    \label{algo:basenintegerlist-remove-tail}

    \begin{algorithmic}
        \Function{removeTail}{\(l\) : BaseNIntegerList} : BaseNIntegerList
            \If{isEmpty(l)}
                \State removeTail \(\leftarrow l\)
            \EndIf

            \State BaseNIntegerElement old\_el \(\leftarrow\) tail(l)
            \State tail(l) \(\leftarrow\) prev(tail(l))
            \State deleteBaseNIntegerElement(old\_el)
            \State size(\(l\)) \(\leftarrow\) size(\(l\)) - 1

            \If{tail(l) = undefined}
                \State head(l) \(\leftarrow\) undefined
            \Else
                \State next(tail(l)) \(\leftarrow\) undefined
            \EndIf

            \State removeTail \(\leftarrow l\)
        \EndFunction
    \end{algorithmic}
\end{algorithm}

\separator

\emph{deleteBaseNIntegerElement}: \emph{BaseNIntegerElement} \(\rightarrow\) \(\emptyset\),
free from the memory the specified BaseNIntegerElement.
\ref{algo:basenintegerlist-delete-basen-integer-element} is the corresponding algorithm.

\begin{algorithm}[H]
    \caption{deleteBaseNIntegerElement algorithm}
    \label{algo:basenintegerlist-delete-basen-integer-element}

    \begin{algorithmic}
        \Function{deleteBaseNIntegerElement}{\(el\) : BaseNIntegerElement} : \(\emptyset\)
            \State deleteBigInteger(value(el))
            \State free(el)
        \EndFunction
    \end{algorithmic}
\end{algorithm}

\separator

\emph{deleteIntegerList}: \emph{BaseNIntegerList} \(\rightarrow\) \(\emptyset\),
clears and deletes the specified BaseNIntegerList.
\ref{algo:basenintegerlist-delete-integer-list} is the corresponding algorithm.

\emph{Principle}: traverse the entire list and free each element.

\begin{algorithm}[H]
    \caption{deleteIntegerList algorithm}
    \label{algo:basenintegerlist-delete-integer-list}

    \begin{algorithmic}
        \Function{deleteIntegerList}{\(l\) : BaseNIntegerList} : \(\emptyset\)
            \If{not isEmpty(l)}
                \State BaseNIntegerElement old\_el \(\leftarrow\) head(l)
                \State BaseNIntegerElement next\_el

                \While{next(old\_el) \(\neq\) undefined}
                    \State next\_el \(\leftarrow\) next(old\_el)
                    \State deleteBaseNIntegerElement(old\_el)
                    \State old\_el \(\leftarrow\) next\_el
                \EndWhile

                \State deleteBaseNIntegerElement(old\_el)
            \EndIf
        \EndFunction
    \end{algorithmic}
\end{algorithm}

\separator

\emph{sumBaseNIntegers}: \emph{BigInteger} \(×\) \emph{BigInteger} \(×\) \emph{Integer} \(\rightarrow\) \emph{\(Array<Character>\)},
sums two integers defined in the given base (without any conversion).
The third parameter is the integers' base.
\ref{algo:basenintegerlist-sum-basen-integers} is the corresponding algorithm.

\emph{Principle}:
\begin{itemize}
    \item Travere simultaneously the two arrays while summing the values in a third Array.
    \item Add the retainer at the end of the third Array if any.
\end{itemize}

\begin{algorithm}[H]
    \caption{sumBaseNIntegers algorithm}
    \label{algo:basenintegerlist-sum-basen-integers}

    \begin{algorithmic}
        \Function{sumBaseNIntegers}{\(a\) : \emph{BigInteger}, \(b\) : \emph{BigInteger}, base : \emph{Integer}} : \emph{BigInteger}
            \State Integer sizeA \(\leftarrow\) size(a)
            \State Integer sizeB \(\leftarrow\) size(b)
            \State Integer sizeC \(\leftarrow\) max(sizeA, sizeB)
            \State \(Array<Character>\) array \(\leftarrow createArray<Character>(sizeC)\)

            \State Integer remainder \(\leftarrow\) 0
            \State Integer i \(\leftarrow\) sizeA - 1
            \State Integer j \(\leftarrow\) sizeB - 1
            \State Integer k \(\leftarrow\) sizeC - 1
            \While{i \(≥\) 0 and j \(≥\) 0}
                \State array\(_k \leftarrow value(a)_i\) + \(value(b)_j\) + remainder
                \State remainder \(\leftarrow array_k\) / base
                \State array\(_k \leftarrow array_k\) mod base

                \State i \(\leftarrow\) i - 1
                \State j \(\leftarrow\) j - 1
                \State k \(\leftarrow\) k - 1
            \EndWhile

            \While{i \(≥\) 0}
                \State array\(_k \leftarrow value(a)_i\) + remainder
                \State remainder \(\leftarrow array_k\) / base
                \State array\(_k \leftarrow array_k\) mod base

                \State i \(\leftarrow\) i - 1
                \State k \(\leftarrow\) k - 1
            \EndWhile

            \While{j \(≥\) 0}
                \State array\(_k \leftarrow value(b)_j\) + remainder
                \State remainder \(\leftarrow array_k\) / base
                \State array\(_k \leftarrow array_k\) mod base

                \State j \(\leftarrow\) j - 1
                \State k \(\leftarrow\) k - 1
            \EndWhile

            \If{remainder \(>\) 0}
                \State sizeC \(\leftarrow\) sizeC + 1
                \State temp\_array \(\leftarrow createArray<Character>(sizeC)\)
                \State temp\_array\(_0 \leftarrow\) remainder
                \State memcpy(temp\_array + 1, array, (sizeC - 1) \(×\) sizeof(Character))
                \State free(array)
                \State array \(\leftarrow\) temp\_array
            \EndIf

            \State sumBinary \(\leftarrow\) createBigInteger(array, sizeC)
        \EndFunction
    \end{algorithmic}
\end{algorithm}

Assumption: \emph{max} is a function that returns the maximum value between the given two.
\emph{sizeof} returns the size taken in memory by the given type.

\separator

\emph{sumIntegerList}: \emph{BaseNIntegerList} \(\rightarrow\) \emph{BigInteger},
sums all the integers defined in the specified list using the binary addition and returns
the corresponding results as an integer (\emph{BigInteger}) represented in the base of the list.
\ref{algo:basenintegerlist-sum-integer-list} is the corresponding algorithm.

\emph{Principle}:
\begin{itemize}
    \item Create an number equal to zero.
    \item Traverse the list while summing each element to the firt number.
\end{itemize}

\begin{algorithm}[H]
    \caption{sumIntegerList algorithm}
    \label{algo:basenintegerlist-sum-integer-list}

    \begin{algorithmic}
        \Function{sumIntegerList}{\(l\) : BaseNIntegerList} : \emph{BigInteger}
            \State array \(\leftarrow createArray<Character>(1)\)
            \State array\(_0 \leftarrow\) 0
            \State sum \(\leftarrow\) createBigInteger(array, 1)

            \State BaseNIntegerElement el \(\leftarrow\) head(l)
            \While{el \(\neq\) undefined}
                \State old\_int \(\leftarrow\) sum
                \State sum \(\leftarrow\) sumBaseNIntegers(sum, value(el), base(l))
                \State deleteBigInteger(old\_int) // avoid memory leak.
                \State el \(\leftarrow\) next(el)
            \EndWhile

            \State sumIntegerList \(\leftarrow\) sum
        \EndFunction
    \end{algorithmic}
\end{algorithm}

\separator

\emph{baseNToDecimal}: \emph{BigInteger} \(×\) \emph{Integer} \(\rightarrow\) \emph{Integer},
converts the specified integer (\emph{BigInteger} represented with the specified base (Integer, second parameter)) into a corresponding integer in base 10.
\ref{algo:basenintegerlist-base-to-decimal} is the corresponding algorithm.

\emph{Principle}
\begin{itemize}
    \item Compute \(\sum\limits_{i=0}^{N - 1}{d_i \times K^i}\), where K is the base and \(d_i)\) the digit in position i.
\end{itemize}

\begin{algorithm}[H]
    \caption{baseNToDecimal algorithm}
    \label{algo:basenintegerlist-base-to-decimal}

    \begin{algorithmic}
        \Function{baseNToDecimal}{n : \emph{BigInteger}, base : \emph{Integer}} : \emph{Integer}
            \State number \(\leftarrow\) 0
            \State i \(\leftarrow\) size(n) - 1
            \State maxWeight \(\leftarrow\) size(n) - 1
            \While{i \(>\) 0}
                \State number \(\leftarrow\) number + value(n)\(_i\) * \(base^{maxWeight - i}\)
                \State i \(\leftarrow\) i - 1
            \EndWhile
            \State baseNToDecimal \(\leftarrow\) number
        \EndFunction
    \end{algorithmic}
\end{algorithm}

\separator

\emph{decimalToBaseN}: \emph{Integer} \(\rightarrow\) \emph{BigInteger},
converts the specified integer represented in base 10 into a corresponding integer (\(Array<Character>\)) in base K.
\ref{algo:basenintegerlist-decimal-to-base} is the corresponding algorithm.

\emph{Principle}
\begin{itemize}
    \item Compute the size of the array.
    \item Create the array.
    \item Compute the number in base K using division by K.
\end{itemize}

\begin{algorithm}[H]
    \caption{decimalToBaseN algorithm}
    \label{algo:basenintegerlist-decimal-to-base}

    \begin{algorithmic}
        \Function{decimalToBaseN}{n : \emph{Integer}, base : \emph{Integer}} : \emph{BigInteger}
            \State arraySize \(\leftarrow\) 1
            \State k \(\leftarrow\) base
            \While{n \(≥\) k}
                \State k \(\leftarrow\) base * k;
                \State arraySize = arraySize + 1;
            \EndWhile
            \State array \(\leftarrow createArray<Character>(arraySize)\)
            \State i \(\leftarrow\) arraySize - 1
            \Repeat
                \State \(array_{i} \leftarrow \) n mod base
                \State n \(\leftarrow\) n / base
                \State i \(\leftarrow\) i - 1;
            \Until{i \(>=\) 0}

            \State \emph{BigInteger} result \(\leftarrow\) createBigInteger(array, arraySize)

            \State decimalToBaseN \(\leftarrow\) result
        \EndFunction
    \end{algorithmic}
\end{algorithm}

\separator

\begin{algorithm}[H]
    \caption{convertBaseToBinary algorithm}
    \label{algo:basenintegerlist-convert-base-to-binary}

    \begin{algorithmic}
        \Function{convertBaseToBinary}{n : \emph{BigInteger}, base : \emph{Integer}} : \emph{BigInteger}
            \State convertBaseToBinary \(\leftarrow\) decimalToBaseN(baseNToDecimal(n, base), 2)
        \EndFunction
    \end{algorithmic}
\end{algorithm}

\separator

\begin{algorithm}[H]
    \caption{convertBinaryToBase algorithm}
    \label{algo:basenintegerlist-convert-binary-to-base}

    \begin{algorithmic}
        \Function{convertBinaryToBase}{n : \emph{BigInteger}, base : \emph{Integer}} : \emph{BigInteger}
            \State convertBinaryToBase \(\leftarrow\) decimalToBaseN(baseNToDecimal(n, 2), base)
        \EndFunction
    \end{algorithmic}
\end{algorithm}

\section{BaseNIntegerListOfList}

\emph{createBucketList}: \emph{Integer} \(\rightarrow\) \emph{BaseNIntegerListOfList},
creates a \emph{BaseNIntegerListOfList} for storing list of integers in base N (N being the specified integer, first parameter).
\ref{algo:basenintegerlistoflist-create-bucket-list} is the corresponding algorithm.

\begin{algorithm}[H]
    \caption{createBucketList algorithm}
    \label{algo:basenintegerlistoflist-create-bucket-list}

    \begin{algorithmic}
        \Function{createBucketList}{base : \emph{Integer}} : BaseNIntegerListOfList
            \State bucketList \(\leftarrow\) create(BaseNIntegerListOfList)

            \State base(bucketList) \(\leftarrow\) base

            \State array(bucketList) \(\leftarrow createArray<BaseNIntegerList>\)(base)
            \For{i from 0 to base - 1}
                \State array(bucketList)\(_i\) \(\leftarrow\) createIntegerList(base)
            \EndFor

            \State createBucketList \(\leftarrow\) bucketList
        \EndFunction
    \end{algorithmic}
\end{algorithm}

\separator

\emph{addIntegerIntoBucket}: \emph{BaseNIntegerListOfList} \(×\) \emph{BigInteger} \(×\) \emph{Integer} \(\rightarrow\) \emph{BaseNIntegerListOfList},
adds a new integer (\emph{BigInteger}) at the end of the specified list in bucket index (index being the third parameter between 0 and base of the integers less 1).
\ref{algo:basenintegerlistoflist-add-integer-into-bucket} is the corresponding algorithm.

\begin{algorithm}[H]
    \caption{addIntegerIntoBucket algorithm}
    \label{algo:basenintegerlistoflist-add-integer-into-bucket}

    \begin{algorithmic}
        \Function{addIntegerIntoBucket}{lol : \emph{BaseNIntegerListOfList}, v : \emph{BigInteger}, index : \emph{Integer}} : \emph{BaseNIntegerListOfList}
            \State // assert(\(index < base(lol)\))
            \State array(lol)\(_{index}\) \(\leftarrow\) insertTail(array(lol)\(_{index}\), v)
            \State addIntegerIntoBucket \(\leftarrow\) lol
        \EndFunction
    \end{algorithmic}
\end{algorithm}

\separator

\emph{buildBucketList}: \emph{BaseNIntegerList} \(×\) \emph{Integer} \(\rightarrow\) \emph{BaseNIntegerListOfList},
builds a new \emph{BaseNIntegerListOfList} according to the specified \emph{BaseNIntegerList} and considering
the specified digit position (second parameter) (rightmost).
\ref{algo:basenintegerlistoflist-build-bucket-list} is the corresponding algorithm.

\begin{algorithm}[H]
    \caption{buildBucketList algorithm}
    \label{algo:basenintegerlistoflist-build-bucket-list}

    \begin{algorithmic}
        \Function{buildBucketList}{l : \emph{BaseNIntegerList}, position : \emph{Integer}} : \emph{BaseNIntegerListOfList}
            \State lol \(\leftarrow\) createBucketList(base(l))
            \State el \(\leftarrow\) head(l)
            \While{el != undefined}
                \State bucketIndex \(\leftarrow\) 0
                \If{position \(<\) size(value(el))}
                    \State bucketIndex \(\leftarrow\) value(value(el))\(_{size(value(el)) - position - 1}\)
                \EndIf

                \State lol \(\leftarrow\) addIntegerIntoBucket(lol, copyBigInteger(value(el)), bucketIndex)
                \State el \(\leftarrow\) next(el)
            \EndWhile

            \State buildBucketList \(\leftarrow\) lol
        \EndFunction
    \end{algorithmic}
\end{algorithm}

\separator

\emph{buildIntegerList}: \emph{BaseNIntegerListOfList} \(\rightarrow\) \emph{BaseNIntegerList},
builds a new \emph{BaseNIntegerList} from the specified \emph{BaseNIntegerListOfList}
(respecting the ascending order on the buckets).
\ref{algo:basenintegerlistoflist-build-integer-list} is the corresponding algorithm.

\begin{algorithm}[H]
    \caption{buildIntegerList algorithm}
    \label{algo:basenintegerlistoflist-build-integer-list}

    \begin{algorithmic}
        \Function{buildIntegerList}{lol : \emph{BaseNIntegerListOfList}} : \emph{BaseNIntegerList}
            \State l \(\leftarrow\) createIntegerList(base(lol))

            \For{i from 0 to base(lol) - 1}
                \State el \(\leftarrow\) head(array(lol)\(_i\))
                \While{el != undefined}
                \State l \(\leftarrow\) insertTail(l, copyBigInteger(value(el)))
                    \State el \(\leftarrow\) next(el)
                \EndWhile
            \EndFor

            \State buildIntegerList \(\leftarrow\) l
        \EndFunction
    \end{algorithmic}
\end{algorithm}

\separator

\emph{deleteBucketList}: \emph{BaseNIntegerListOfList} \(\rightarrow \emptyset\),
clears and deletes the specified \emph{BaseNIntegerListOfList} (free the previously allocated
memory).
\ref{algo:basenintegerlistoflist-delete-bucket-list} is the corresponding algorithm.

\begin{algorithm}[H]
    \caption{deleteBucketList algorithm}
    \label{algo:basenintegerlistoflist-delete-bucket-list}

    \begin{algorithmic}
        \Function{deleteBucketList}{lol : \emph{BaseNIntegerListOfList}} : \(\emptyset\)
            \For{i from 0 to base(lol) - 1}
                \State deleteIntegerList(array(lol)\(_i\))
            \EndFor
            \State free(array(lol))
        \EndFunction
    \end{algorithmic}
\end{algorithm}

\section{Radix sort}

\emph{radixSort}: \emph{BaseNIntegerList} \(\rightarrow\) \emph{BaseNIntegerList},
sorts the specified \emph{BaseNIntegerList} using radix sort.
\ref{algo:basenintegerlistoflist-radix-sort} is the corresponding algorithm.

Principle: …

\begin{algorithm}[H]
    \caption{radixSort algorithm}
    \label{algo:basenintegerlistoflist-radix-sort}

    \begin{algorithmic}
        \Function{radixSort}{l : \emph{BaseNIntegerList}} : \emph{BaseNIntegerList}
            \State maxIterations \(\leftarrow\) 0
            \State el \(\leftarrow\) head(l)
            \While{el != undefined}
                \State maxIterations \(\leftarrow\) max(maxIterations, size(value(el)) - 1)
                \State el \(\leftarrow\) next(el)
            \EndWhile

            \State maxIterations \(\leftarrow\) maxIterations + 1

            \For{i from 0 to maxIterations}
                \State bucketList \(\leftarrow\) buildBucketList(l, i)
                \State deleteIntegerList(l)
                \State l \(\leftarrow\) buildIntegerList(bucketList)
                \State deleteBucketList(bucketList)
            \EndFor

            \State radixSort \(\leftarrow\) l
        \EndFunction
    \end{algorithmic}
\end{algorithm}

\chapter{Summary / conclusion}

…

%##################### Bibliography ########################

\bibpunct{[}{]}{,}{a}{}{;}
\bibliographystyle{plainnat}
\bibliography{bibliography}{}

\end{document}

